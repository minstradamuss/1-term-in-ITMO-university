documentclass{article}
usepackage[utf8]{inputenc}
usepackage[T2A]{fontenc}
usepackage[russian]{babel}
usepackage{amsfonts}
usepackage{amsmath}
usepackage{amssymb}
usepackage{arcs}
usepackage{fancyhdr}
usepackage{float}
usepackage[left=3cm,right=3cm,top=2.5cm,bottom=3cm]{geometry}
usepackage{graphicx}
usepackage{hyperref}
usepackage{multicol}
usepackage{stackrel}
usepackage{xcolor}


begin{document}
pagestyle{empty}
normalsize

section{Матанализ от Виноградова}
label{sec}
subsection{}
label{subsec1.1}

noindent Норма в пространстве $L_{p}(E, mu)$ равна

begin{center}
$ f _{L_{p}(E, mu))}=left (displaystyle {int_E}left  f right ^{p}dmu  right )^{frac{1}{p}}$
end{center}


Положительная однородность очевидна, неравенству треугольника соответствует неравенство Минковского, а из $left  f right $ следует $fsim 0$, то есть $f=0$ как элемент $L_{p}(E, mu)$.

Пусть $(X, mathbb{A}, mu)$ — пространство с мерой, $E in mathbb{A}$. Полагают


begin{center}
$L_{infty} (E, mu) = 
begin{Bmatrix}
 f text{п.в.} E rightarrow overline{mathbb{R}} left ( textup{или}overline{mathbb{C}} right ),  
 ftext{измерима},  
 operatorname{ess}  suplimits_{x in E}  left f(x) right  + infty
end{Bmatrix}$
end{center}


Эквивалентные функции отождествляются. Легко видеть, что $L_{infty}(E, mu)$ — векторное пространство. Норма в $L_{infty}(E, mu)$ задается равенством


begin{center}
$left  f right _{L_{infty}(E, mu)} = operatorname{ess}  suplimits_{x in E}
 left f(x) right $
end{center}
subsection{}
label{subsec1.2}
noindent Непустое семейство $mathbb{A}$ подмножеств $X$ называется $sigma$-textbf{алгеброй}, если выполняются следующие два условия.

begin{enumerate}
  item Если $A in mathbb{A}$, то $A^{c} in mathbb{A}$.
  item Если $A_{k} in mathbb{A}$ при всех $k in N$, то $displaystyle bigcup_{k=1}^{infty} A_{k} in mathbb{A}$.
end{enumerate}


noindent Свойства 1 и 2 называются аксиомами $sigma$-алгебры.
section{Большое задание от доктора Тренча}
subsection{}
label{subsec2.1}


noindent Let $y=ue^{3x}$. Then 


begin{align}
{y}''-3{y}'+2y &= e^{3x}[({u}''+6{u}'+9u)-3({u}'+3u)+2u] 
&= e^{3x}({u}''+3{u}'+2u) = e^{3x}[21cos x-(11+10x)sin x]
end{align}

noindent 
if $displaystyle {u}''+3u'+2u=21cos x-(11 + 10x)sin x$. Now let

begin{center}
  begin{tabular}{rcl}
    $u_{p}$&=&$(A_{0}+A_{1}x)cos x + (B_{0}+B_{1}x)sin x; text{ then}$ 
    ${u}'_{p}$&=&$(A_{1}+B_{0}+B_{1}x)cos x + (B_{1}-A_{0}-A_{1}x)sin x$
    ${u}''_{p}$&=&$(2B_{1}-A_{0}-A_{1}x)cos x - (2A_{1}+B_{0}+B_{1}x)sin x, text{ so}$
  end{tabular}
end{center}
%
begin{center}
begin{tabular}{rcl}
${u}''+3{u}'+2u$ & = & $[A_{0}+3A_{1}+3B_{0}+2B_{1}+(A_{1}+3B_{1})x]cos x$ 
 &  & $ + [B_{0}+3B_{1}-3A_{0}-2A_{1}+(B_{1}-3A_{1})x]sin x$  
 & = & $21cos x-(11+10x)sin x  text{if}$ 
end{tabular}
end{center}


begin{tabular}{rcr}
$A_1 + 3B_1$ & = & 0 
$-3A_1 + B_1$ & = & $-10$
end{tabular}
and
begin{tabular}{rcr}
$A_0 + 3B_0 + 3A_1 + 2B_1$ & = & 21 
$-3A_0 + B_0 - 2A_1 + 3B_1$ & = & $-11$
end{tabular}.


 

noindent From the first two equations $A_{1} = 3$, $B_{1} = -1$. Substituting these in last two equations yields and solving for $A_{0}$ and $B_{0}$ yields $A_{0} = 2$, $B_{0} = 4$. Therefore, $u_{p} = (2 + 3x) cos x + (4 - x) sin x$ and $y_{p} = e^{3x}[(2 + 3x) cos x + (4 - x) sin x]$. The characteristic polynomial of the complementary equation is $p(r) = r^{2}-3r+2 = (r-1)(r-2)$, so ${e^{x}, e^{2x}}$ is a fundamental set of solutions of the complementary equation, and (A) $y = e^{3x}[(2 + 3x) cos x + (4 - x) sin x] + c_{1}e^{x} + c_{2}e^{2x}$ is the general solution of the nonhomogeneous equation. Differentiating (A) yields

begin{center}
begin{tabular}{rcl}
	${y}'$ & = & $ 3e^{3x}[(2 + 3x) cos x + (4 - x) sin x]$  
	&         & $+e^{3x}[(7 - x) cos x - (3 + 3x) sin x] + c_{1}e^{x} + 2c_{2}e^{2x}$.
end{tabular}
end{center}

noindent Therefore, $y(0)=0,{y}'(0)=6Rightarrow 0=2+c_{1}+c_{2}, 6=6+7+c_{1}+2c_{2}, text{so} c_{1}+c_{2}=2,  c_{1}+2c_{2}=-7.$
Therefore, $c_{1}=3,  c_{2}=-5,  text{and } y=e^{3x}[(2+3x)cos x+(4-x)sin x]+3e^{x}-5e^{2x}.$

section{Маленькие задания от доктора Тренча}

subsection{}
label{subsec3.1}
noindent $sinh at leftrightarrow displaystyle frac{a}{s^{2}-a^{2}} text{ and }  cosh at leftrightarrow frac{1}{s^{2}-a^{2}}, text{ so } H(s)=frac{as}{(s^{2}-a^{2})^{2}}.$
subsection{}
label{subsec3.2}
noindent $tsinomega t leftrightarrow displaystyle frac{2omega s}{(s^{2}+omega^{2})^{2}} text{ and }  tcosomega t leftrightarrow frac{s^{2}-omega^{2}}{(s^{2}+omega^{2})^{2}}, text{ so }  H(s)=frac{2omega s(s^{2}-omega^{2})}{(s^{2}+omega^{2})^{4}}.$
subsection{}
label{subsec3.3}

noindent $displaystyle e^{t}leftrightarrow frac{1}{s-1} text{ and } sin at leftrightarrow displaystyle frac{a}{s^{2}+a^{2}}, text{ so } H(s)=frac{a}{(s-1)(s^{2}+a^{2})}.$

end{document}