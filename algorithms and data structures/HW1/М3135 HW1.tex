\documentclass{homework}
\usepackage[T2A]{fontenc}
\usepackage[utf8]{inputenc}
\usepackage[russian]{babel}
\parindent 0pt
\parskip 8pt

\title{M3135 HW1 Лоскутова Мария}

\begin{document}
\maketitle

\section{}
\label{sec:1}
 \begin{gather*}  
	1) \
        f_1(n) = \mathcal{O}(g_1(n)) \, \Rightarrow \, f_1(n) \, \leq \, c_1 * g_1(n)
	\end{gather*}
 \begin{gather*}  
	2) \
        f_2(n) = \mathcal{O}(g_2(n)) \, \Rightarrow \, f_2(n) \, \leq \, c_2 * g_2(n)
	\end{gather*}
 \begin{gather*}  
	3) \
        c_3 = max(c_1, c_2)
	\end{gather*}
\begin{gather*}  
        f_1(n) + f_2(n) \, \leq \, c_3 * g_1(n) + c_3 * g_2(n) \, \Rightarrow \,
        f_1(n) + f_2(n) \, \leq \, c_3 * (g_1(n) + g_2(n)) = \mathcal{O}(g_1(n) + g_2(n))
        \text{ - Q.E.D.}
	\end{gather*}

\section{} 
\label{sec:2}
\begin{gather*}
	max(f(n), g(n)) = \Theta(f(n) + g(n))
	\\
	\Updownarrow
	\\
	\begin{cases}
		max(f(n), g(n)) = \mathcal{O}(f(n) + g(n))
		\\
		max(f(n), g(n)) = \Omega(f(n) + g(n))
	\end{cases}
	\end{gather*}
1) \ 1. Рассмотрим первую часть системы:
	\begin{gather*}
        max(f(n), g(n)) = \mathcal{O}(f(n) + g(n)) \, \Rightarrow \, max(f(n), g(n)) \, \leq \, c1 * (f(n) + g(n)) \
        \end{gather*}  
        2. \
        пусть 
        \begin{gather*}
        N_0 = 1, c_1 = 1
        \end{gather*}
        тогда,
        \begin{gather*}
        max(f(n), g(n)) \, \leq \, f(n) + g(n)
        \end{gather*}
Докажем это неравенство. Рассмотрим произвольное 
\begin{gather*}
        N^{\prime} > n_0 \
        \end{gather*}
Педположим, что
\begin{gather*}
        f(N^{\prime}) \, \geq \, g(N^{\prime}) \
        \end{gather*}
тогда,
\begin{gather*}
        max(f(N^{\prime}), g(N^{\prime})) \, = \, f(N^{\prime}) \, \Rightarrow \, f(N^{\prime}) \, \leq \, f(N^{\prime}) + g(N^{\prime}) \text{ - верно,}
        \end{gather*}
иначе аналогично.
\


2) \ 1. Рассмотрим вторую часть системы:
	\begin{gather*}
        max(f(n), g(n)) = \Omega(f(n) + g(n)) \, \Rightarrow \, max(f(n), g(n)) \, \geq \, c_2 * (f(n) + g(n)) \
        \end{gather*}  
        2. \
        пусть 
        \begin{gather*}
        N_0 = 1, c_2 = 0.5
        \end{gather*}
        тогда,
        \begin{gather*}
        max(f(n), g(n)) \, \geq \, 0.5 * f(n) + 0.5 * g(n)
        \end{gather*}

Докажем это неравенство. Рассмотрим произвольное 
\begin{gather*}
        N^{\prime} > n_0 \
        \end{gather*}
Педположим, что
\begin{gather*}
        f(N^{\prime}) \, \geq \, g(N^{\prime}) \
        \end{gather*}
тогда,
\begin{gather*}
        max(f(N^{\prime}), g(N^{\prime})) \, = \, f(N^{\prime}) \, \Rightarrow \, f(N^{\prime}) \, \geq \, 0.5 * f(N^{\prime}) + 0.5 * g(N^{\prime}) \text{ - верно,}
        \end{gather*}
иначе аналогично.

\section{} 
\label{sec:3}
 \begin{gather*}
		\displaystyle\sum_{i = 1}^{n+5} 2^i = \mathcal{O}(2^n)
		\end{gather*}
1) Используем формулу суммы геометрической прогрессии:  
  \begin{gather*}  
        S = \frac{b_1*(q^{n+1})}{q - 1}
	\end{gather*}
 Таким образом,
 \begin{gather*}  
        S = \frac{2*(2^{n+6})}{1} = 2^{n+7} - 2 = 128 * 2 ^ n - 2 = \mathcal{O}(2^n) \text{ - Q.E.D.}
	\end{gather*}
  

\section{} 
\label{sec:4}
	\begin{gather*}
		\frac{n^3}{6} - 7n^2 = \Omega(n^3)
		\\
		\Updownarrow
		\\
		\exists \, c>0,\, \exists N>0,\: \forall n > N:\: \frac{n^3}{6} - 7n^2 \geq c\cdot n^3
		\end{gather*}
Пусть:
		\begin{gather*}
		c = \frac{1}{12},\: N = 84
		\end{gather*}
Тогда, 
            \begin{gather*}
		\frac{n^3}{6} - 7n^2 \geq \frac{1}{12}\cdot n^3  \, \Rightarrow \,
		n^3 - 84n^2 \geq 0
		 \, \Rightarrow \,
		n - 84 \geq 0 
		 \, \Rightarrow \,
		n \geq 84\text{ - верно.}
	\end{gather*}



\section{}
\label{sec:5}
	\begin{gather*}
		1
		\text{\hyperref[p1]{\vphantom{|}$\longrightarrow$}}
		\,
		\left(\frac{3}{2}\right)^2
		\,
		\text{\hyperref[p2]{\vphantom{|}$\longrightarrow$}}
		\,
		n^{\frac{1}{\log n}}
		\,
		\text{\hyperref[p3]{\vphantom{|}$\longrightarrow$}}
		\,
		\log\log n
		\,
		\text{\hyperref[p4]{\vphantom{|}$\longrightarrow$}}
		\,
		\sqrt{\log n}
		\,
		\text{\hyperref[p5]{\vphantom{|}$\longrightarrow$}}
		\,
		\log^2n
		\,
		\text{\hyperref[p6]{\vphantom{|}$\longrightarrow$}}
		\,
		(\sqrt 2)^{\log n}
		\,
		\text{\hyperref[p7]{\vphantom{|}$\longrightarrow$}}
		\,
		2^{\log n}
		\,
		\text{\hyperref[p8]{\vphantom{|}$\longrightarrow$}}
		\,
		n
		\,
		\text{\hyperref[p9]{\vphantom{|}$\longrightarrow$}}
		\,
		n\log n
		\,
		\text{\hyperref[p10]{\vphantom{|}$\longrightarrow$}}
		\,
		\log n!
		\,
		\text{\hyperref[p11]{\vphantom{|}$\longrightarrow$}}
		\\
		n^2
		\,
		\text{\hyperref[p12]{\vphantom{|}$\longrightarrow$}}
		\,
		4^{\log n}
		\,
		\text{\hyperref[p13]{\vphantom{|}$\longrightarrow$}}
		\,
		n^3
		\,
		\text{\hyperref[p14]{\vphantom{|}$\longrightarrow$}}
		\,
		\left(\log n\right)!
		\,
		\text{\hyperref[p15]{\vphantom{|}$\longrightarrow$}}
		\,
		\left(\log n\right)^{\log n}
		\,
		\text{\hyperref[p16]{\vphantom{|}$\longrightarrow$}}
		\,
		n^{\log\log n}
		\,
		\text{\hyperref[p17]{\vphantom{|}$\longrightarrow$}}
		\,
		n\cdot 2^n
		\,
		\text{\hyperref[p18]{\vphantom{|}$\longrightarrow$}}
		\,
		e^n
		\,
		\text{\hyperref[p19]{\vphantom{|}$\longrightarrow$}}
		\,
		n!
		\,
		\text{\hyperref[p20]{\vphantom{|}$\longrightarrow$}}
		\,
		\left(n+1\right)!
		\,
		\text{\hyperref[p21]{\vphantom{|}$\longrightarrow$}}
		\,
		2^{2^n}
		\,
		\text{\hyperref[p22]{\vphantom{|}$\longrightarrow$}}
		\,
		2^{2^{n+1}}
	\end{gather*}
\end{document}
